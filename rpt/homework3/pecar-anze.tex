% Homework report template for courses lectured by Blaz Zupan.
% For more on LaTeX please consult its documentation pages, or
% read tutorials like http://tobi.oetiker.ch/lshort/lshort.pdf.
%
% Use pdflatex to produce a PDF of a report.

\documentclass[a4paper,11pt]{article}
\usepackage{a4wide}
\usepackage{fullpage}
\usepackage[utf8x]{inputenc}
\usepackage[toc,page]{appendix}
\usepackage[pdftex]{graphicx} % for figures
\usepackage{setspace}
\usepackage{color}
\definecolor{light-gray}{gray}{0.95}
\usepackage{listings} % for inclusion of Python code
\usepackage{hyperref}
\renewcommand{\baselinestretch}{1.2}

\lstset{ % style for Python code, improve if needed
language=Python,
basicstyle=\footnotesize,
basicstyle=\ttfamily\footnotesize\setstretch{1},
backgroundcolor=\color{light-gray},
}

\title{Homework \#3: Evolutionary Tree}
\author{Anže Pečar (63060257)}
\date{\today}

\begin{document}

\maketitle

\section{Introduction}

In the third homework we compare COX3 mitochindrial gene between multiple species and then build an evolutionary tree.

\section{Data}

Data was obtained from GenBank. We downloaded mitochondrial refrence sequences (RefSeq) for Gray Wolf (NC\_002008.4), Goldfish (NC\_006580.1), Chameleon (NC\_012420.1), Daboia (NC\_011391.1), Dolphin (NC\_012061.1), Horse (NC\_001640.1), Gorilla (NC\_001645.1), Human (NC\_012920.1), Neanderthal (NC\_011137.1), Chimpanzee (NC\_001643.1), Orangutan (NC\_002083.1), Rat (NC\_001665.2), Boar (NC\_014692.1) and Pufferfish (NC\_004299.1). We used common English names instead of Latin ones, because they are easier to understand.


\section{Methods}

We implemented the Needleman-Wunsch algorithm which performs a global alignment on two sequences. The algorithm is implemented using dynamic programming. 

We used the $BLOSUM50$ amino acid substitution matrix. This is the default scoring used by FASTA and TFASTA for comparison of amino acid sequences.

Choosing different gap penalties didn't produce a significantly different result. As an example if we chose gap penalty of -50 instead of -5, most of the dendrogram remained and only one or two species switched positions (Neanderthal got pushed up a few levels). A gap penalty of -5 seemed to provide us with the best results.
\section{Results}



\subsection{Table of alignment scores}
Table \ref{scores} lists pairwise alignment scores for all the species.
\begin{table}[htbp,resetmargins=true]
\footnotesize
\caption{Alignment scores}
\label{scores}
\begin{center}
\begin{tabular}{@{}lllllllllllllll@{}}
\hline
 & 1.  & 2.  & 3.  & 4.  & 5.  & 6.  & 7.  & 8.  & 9.  & 10.  & 11.  & 12.  & 13.  & 14.\\
\hline

1. Gray Wolf & 1832 & 1556 & 1298 & 1381 & 1682 & 1688 & 1577 & 1608 & 1591 & 1603 & 1574 & 1672 & 1671 & 1555\\
2. Goldfish & 1556 & 1824 & 1305 & 1417 & 1553 & 1564 & 1583 & 1585 & 1580 & 1600 & 1551 & 1552 & 1581 & 1743\\
3. Veiled Chameleon & 1298 & 1305 & 1827 & 1290 & 1329 & 1312 & 1311 & 1304 & 1289 & 1296 & 1273 & 1322 & 1307 & 1305\\
4. Daboia & 1381 & 1417 & 1290 & 1796 & 1395 & 1410 & 1422 & 1406 & 1401 & 1409 & 1386 & 1405 & 1400 & 1414\\
5. Dolphin & 1682 & 1553 & 1329 & 1395 & 1826 & 1685 & 1612 & 1640 & 1621 & 1622 & 1585 & 1628 & 1704 & 1547\\
6. Horse & 1688 & 1564 & 1312 & 1410 & 1685 & 1815 & 1608 & 1616 & 1610 & 1606 & 1585 & 1658 & 1717 & 1556\\
7. Gorilla & 1577 & 1583 & 1311 & 1422 & 1612 & 1608 & 1814 & 1764 & 1759 & 1757 & 1727 & 1618 & 1602 & 1542\\
8. Human & 1608 & 1585 & 1304 & 1406 & 1640 & 1616 & 1764 & 1823 & 1804 & 1777 & 1713 & 1625 & 1621 & 1550\\
9. Neanderthal & 1591 & 1580 & 1289 & 1401 & 1621 & 1610 & 1759 & 1804 & 1816 & 1772 & 1708 & 1620 & 1602 & 1541\\
10. Chimpanzee & 1603 & 1600 & 1296 & 1409 & 1622 & 1606 & 1757 & 1777 & 1772 & 1820 & 1703 & 1643 & 1612 & 1564\\
11. Orangutan & 1574 & 1551 & 1273 & 1386 & 1585 & 1585 & 1727 & 1713 & 1708 & 1703 & 1804 & 1593 & 1588 & 1531\\
12. Rat & 1672 & 1552 & 1322 & 1405 & 1628 & 1658 & 1618 & 1625 & 1620 & 1643 & 1593 & 1816 & 1645 & 1545\\
13. Boar & 1671 & 1581 & 1307 & 1400 & 1704 & 1717 & 1602 & 1621 & 1602 & 1612 & 1588 & 1645 & 1814 & 1573\\
14. Pufferfish & 1555 & 1743 & 1305 & 1414 & 1547 & 1556 & 1542 & 1550 & 1541 & 1564 & 1531 & 1545 & 1573 & 1826\\

\hline
\end{tabular}
\end{center}
\end{table}

\subsection{The dendrogram}

Figure \ref{dendro} shows the calculated evolutionary tree. We used average linkage to calculate distances. 

\begin{figure}[h!]
\begin{center}
\includegraphics[scale=0.60]{dendro1.png}
\caption{Dendrogram with avrage linkage}
\label{dendro}
\end{center}
\end{figure}

\section*{Honor Code}

% The following paragraph of your report should be included as is - do % not change it.

My answers to homework are my own work. I did not make solutions or code available to anyone else. I did not engage in any other activities that will dishonestly improve my results or dishonestly improve/hurt the results of others.

\end{document}